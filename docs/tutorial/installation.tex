
The Zoltar tool set depends on the Low Level Virtual Machine framework and
is currently only known to be compatible with version 2.6.
This version of \verb|llvm| can be downloaded from 
\verb|http://llvm.org| and can be installed according to their instructions.
Also, make sure the \verb|ncurses| and \verb|libgtk-2.0dev| packages are
installed on your system.
At the time of writing this tutorial, the tools are only tested on linux (Ubuntu and Fedora).

\section{Installing llvm}
The following shows an example of installing the \verb|llvm| tools.
This serves as a guideline to quickly get a working toolset,
but does not cover all details and is not guaranteed to work on your system.
It is recommended to look at the getting-started guide on the \verb|llvm| website:\\
\verb|  http://llvm.org/docs/GettingStarted.html|

First, download the \verb|llvm| version 2.6 sources and the
\verb|llvm-gcc frontend| sources.
The latter is required to compile C programs using \verb|llvm|.
Both are located at\\
\verb|  http://llvm.org/releases/download.html#2.6|\\
Next, unpack both using\\
\verb|  # tar -xzf llvm-2.6.tar.gz|\\
\verb|  # tar -xzf llvm-gcc-4.2-2.6.source.tar.gz|\\
Make an object directory and an install directory for llvm:\\
\verb|  # mkdir llvm-obj|\\
\verb|  # mkdir llvm-install|\\
Install \verb|llvm|:\\
\verb|  # cd llvm-obj|\\
\verb|  # ../llvm-2.6/configure|\\
\verb|  # make|\\
At this point, an \verb|llvm| version without gcc-frontend is built.
To make the gcc-frontend, however, we need this installation of \verb|llvm|.
Make an object directory and an install directory for \verb|llvm-gcc| as well:\\
\verb|  # cd ..|\\
\verb|  # mkdir llvm-gcc-obj|\\
\verb|  # mkdir llvm-gcc-install|\\
and configure it to use the \verb|llvm| we have just installed.\\
\verb|  # cd llvm-gcc-obj|\\
\verb|  # ../llvm-gcc4.2-2.6.source/configure \ |\\
\verb|  > --prefix=`pwd`/../llvm-gcc-install --program-prefix=llvm- \ |\\
\verb|  > --enable-llvm=`pwd`/../llvm-obj --enable-languages=c,c++ |\\
if you are compiling in a 64 bit system, add the following option if during
\verb|make| it fails complaining about \verb|user32.h| headers\\
\verb|  > --disable-multilib |\\
\verb|  # make|\\
\verb|  # make install|\\
After this, \verb|llvm| must be reconfigured to include \verb|llvm-gcc|.\\
\verb|  # cd ../llvm-obj|\\
\verb|  # ../llvm-2.6/configure --prefix=`pwd`/../llvm-install \ |\\
\verb|  > --withllvmgccdir=`pwd`/../llvm-gcc-install|\\
\verb|  # make |\\
\verb|  # make install |

The above procedure will have created two installation directories,
\verb|llvm-install| and \verb|llvm-gcc-install|.
The \verb|bin| directories of both should be in the \verb|PATH| environment,
enabling a system wide call to the \verb|llvm| tools.


\section{Installing Zoltar}

The source package can be downloaded from \verb|http://fdir.org|.
Unpack it with \\
\verb|  # tar -xzf zoltar.tar.gz|\\
Inside the unpacked directory run \\
\verb|  # ./configure --with-llvmsrc=<LLVMSRCDIR> \|\\
\verb|  > --with-llvmobj=<LLVMOBJDIR> --prefix=<INSTALLDIR>|\\
where \verb|<LLVMSRCDIR>| and
\verb|<LLVMOBJDIR>| should be the directory containing the 
\verb|llvm| source files and \verb|llvm| object files respectively
(\verb|/path/to/llvm-2.6| and \verb|/path/to/llvm-obj| in the previous section).
\verb|<INSTALLDIR>| should be the place where you want the tool binaries and libraries to
be located.
Next, build and install the tool set using \\
\verb|  # make|\\
\verb|  # make install|\\
At this point the Zoltar tools are installed in the \verb|bin| directory of \verb|<INSTALLDIR>|
and the libs are installed in the \verb|lib| directory.
Make sure to include the \verb|bin| directory in the \verb|PATH| environment 
and to include the \verb|lib| dir in the library search path (using libtool) or, alternatively,
using the \verb|-L/path/to/lib| option during linking, or
copy the \verb|*.so*| files to a library search path, such as \verb|/usr/lib|
(this is not recommended and is not seen as a general solution, though).
In the remainder of the tutorial it is assumed that the linker can find the libraries.

The resulting libraries and binaries are given in Table \ref{t:LibsAndTools} 
along with a short description.

\begin{table}
  \begin{tabular}{l|l}
    \hline
	\verb|libinstrument| & library which should be linked to the instrumented program \\
	\verb|libpasses|     & library with instrumentation passes loaded by the \verb|instrument| tool \\
	\verb|instrument|    & instrumentation tool \\
	\verb|zoltar|       & console based program analysis and configuring tool \\
	\verb|xzoltar|      & source code visualization of program analysis \\
    \hline
  \end{tabular}
  \caption{Description of libraries and tools}
  \label{t:LibsAndTools}
\end{table}


\section{Development Build}
  
To enable development of the Zoltar tools a debug installation is available.
This will enable debugging context to be compiled within the binaries and libraries,
which will aid in developing and debugging the toolset.
To build the debug version of the Zoltar tool set
also \verb|llvm| needs to be built in debug mode.
To do this, add the \verb|ENABLE_OPTIMIZED=0| option to the \verb|make|
commands while building the final version of \verb|llvm| (with llvm-gcc linked).\\
\verb|  # make ENABLE_OPTIMIZED=0| \\
\verb|  # make install ENABLE_OPTIMIZED=0| \\
This debug install procedure has then be done for the Zoltar toolset as well.
Note that \verb|llvm| has to be built in debug mode before building Zoltar in debug mode,
since the latter requires certain libraries of the debug installation of \verb|llvm|.

Within the directory of each tool or library in the Zoltar package one can type
\verb|make| to build a release version of that particular tool or library or type
\verb|make ENABLE_OPTIMIZED=0| to build a debug version.

The use of \verb|ENABLE_OPTIMIZED=0| is only a temporal flag. If you want to make
debug the default option, use the following flag when compiling llvm and zoltar:\\
\verb|  > --disable-optimized |\\

